\documentclass{article}
\usepackage[utf8]{inputenc}
\usepackage{parskip}
\usepackage{graphicx}
\usepackage{amsmath}
\usepackage{blindtext}

\title{Information Theory for Neural Network Regularization}
\author{Jakub Bartczuk}
\date{June 2020}


\newcommand{\loss}{\mathcal{L}}
\begin{document}
\maketitle

\section{Introduction}

Learning neural networks for supervised learning is commonly formalized as optimizing log loss

$$\loss = E_{(x,y) \sim p}[log\, p_\theta(y|x)] \approx \frac{1}{N} \sum_i log\, p_\theta(y_i|x_i) $$

When model, $p_\theta$ is a complicated function there is risk of overfitting - the model might learn useless features that do not generalize from training set. 

To address that problem, most neural network use some form of \textit{regularization}. Regularization is performed by applying some procedure to neural network outputs or by optimizing $\loss' = \loss + P$ where $P$ is some penalty.

Usual regularization methods are based on injecting noise - be it in explicit form of dropout (zeroing in activations of intermediate layers) or implicit in the form of weight decay (it is a known fact that for linear models L2 weight decay is equivalent to assuming a prior normal distribution on weights which is in turn equivalent to adding gaussian noise).

\section{Information theory and supervised learning}

Log loss has natural information theoretic interpretation: $\loss = - H(p) - KL(p_{\theta}|p)$, so optimizing w.r.t $p_{\theta}$ means minimizing KL divergence between model and true probability. 

It is an empirical fact that when overfitting often manifests in model being overly confident of its predictions. This is because $L$ only measures error for true class $y_i$ and discards information about other classes.

In practice the problem of model being overly confident is commonly solved by using \textit{label smoothing} which turns one-hot encoded vector $y_i$ that represents indicator function for appropriate class into vector which has some nonzero probabilities for other classes. 

\subsection{Regularizing Neural Networks by Penalizing Confident Output Distributions}

Less known method that is based on information-theoretic interpretation is penalizing confident distributions from \cite{penalizing}. Distribution confidence is measured using entropy.In this method becomes $\loss'$

$$\loss' = E_{(x,y) \sim p}[log\, p_\theta(y|x)] + \lambda H(p_\theta(y|x))$$

\section{Information Bottleneck}


\begin{thebibliography}{9}
\bibitem{penalizing}
  Gabriel Pereyra, George Tucker, Jan Chorowski, Łukasz Kaiser, Geoffrey Hinton,
  Leslie Lamport,
  \textit{Regularizing Neural Networks by Penalizing Confident Output Distributions},
  2017.

\end{thebibliography}

\end{document}
